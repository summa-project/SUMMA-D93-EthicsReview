\section{Conclusion}
\label{sec:concl}

The aim of this report is to introduce the key ethical questions and challenges that the \project project must address.  There are three principal challenges which have a direct and immediate relation to our work in \project.  We enumerate them below, together with our current position.

\begin{enumerate}
    \item \textbf{Privacy.}  Privacy is possibly the main ethical issue arising from \project, in particular the work we do on social media analysis.  Privacy will be designed into the \project data management infrastructure, as discussed in the Data Management Plan (D2.1; Section 6).  Going forward we shall prepare a Privacy Impact Assessment for \project, which will be included in the updated version of this report (D9.4, due in M18).
    
    \item \textbf{Sharing data outside of the EU.} A specific issue that arises for \project is sharing data with QCRI (Qatar) who are outside the EU.  (QCRI are a third party linked to UEDIN.)  In particular, sharing social media data with QCRI may be in breach of EU regulations on the export of personal data so we will take account of that when managing project data internally.  External/demonstration systems are less of an issue since our systems may be run entirely within the EU.
    
    \item \textbf{Data management.}  Practical ethical issues relating to data management (for example, (pseudo) anonymisation of social media data) are discussed in the Data Management Plan (D2.1).  Other issues include determining what happens to data after the project ends.  While there are strong privacy reasons to destroy social media data, in particular, we note that the research programmes of the partners will continue after the project, thus this data will continue to have strong research utility.  In addition, we must balance the requirement to respect privacy with the requirement for reproducibility in research -- data destruction also destroys reproducibility.  
\end{enumerate}

We will ensure in particular that the \project Data Management Plan (D2.1) stays well-aligned with the Ethics Review and Recommendations. This includes ensuring that ethical considerations are part of the design process – for example, privacy by design, and making explicit who has access to which data, and what will be done with it. The Data Management Plan shall specifically describe clearly where in the \project architecture we process personal data, how such data is protected, and for what purpose the data is processed.

Ethical issues are an ongoing process, and through dialogue with our external advisors, and other stakeholders -- as well as intensive internal dialogue -- we expect to refine our response to these challenges.  Of course, there is no ``right answer'' since we are continually balancing benefits and risks.  The main goals for the next year will be ongoing review of all ethical issues relating to \project, especially in relation to the Data Management Plan, and the development of a Privacy Impact Assessment.

