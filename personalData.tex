\section{Protection of personal data}
\label{sec:personal}

The main ethical issues arising from \project are to do with the protection of personal data.  This affects the social media that we collect in the project, as well as -- to a more limited extent --  the broadcast audio and video.  We note that social media collection is based on a number of accounts curated by project partners (principally BBC and DW), rather than a large-scale processing of social media.

It is important that, across the project,  we develop an understanding of the impact the technologies we develop may have on people.  This includes assessing the impact of \project data handling on individuals whose tweets and other social media data are used. This risk could extend to ``regular'' broadcast content, where individuals are mentioned.  Following the initial meeting of the EAB, a number of issues were highlighted by the external experts as being of particular importance.
\begin{itemize}
\item A clear description of what research and innovation activities \project data will be used for and who has the responsibility for handling, storing, and destroying the data (data processing).
\item A clear description of the purpose of our research and innovation, in order to make clear that there is a substantial public interest in the  work of the project.
\item A clear description of the safeguards that we shall put in place, such as (pseudo) anonymisation of social media data
\item Identification of the countries in which data will be processed or reside, together with an understanding of the national privacy and data protection regulations, and engagement with the relevant data protections agencies .
\item Transparent presentation of the work of the project to the public (as carried out in WP8) which reiterates that the purpose of our research is not the individuals whose social media postings we are collecting, but the news-worthiness of the content (with the exception of public figures and politicians who are an integral part of news reporting).
\end{itemize}


To address this, and to provide a coherent framework, we plan to develop an ongoing \emph{Privacy Impact Assessment} for \project.  The privacy impact assessment, will include:
\begin{itemize}
    \item A description of the information flows in the project (distilled from the data management plan);
    \item Identification of the privacy and related risks;
    \item Actions taken by \project to reduce the identified risks;
    \item Integration and of these outcomes into the project plan, in particular the Data Management Plan;
    \item A description of how the \project platform may be reused, and who the potential users will be during and after the project.
\end{itemize}
This framework for privacy impact assessment is derived from that developed by the UK Information Commissioner's Office (ICO)\footnote{\url{https://ico.org.uk/media/for-organisations/documents/1595/pia-code-of-practice.pdf}} which was recommended by the EAB.

Based on these analyses there are several actions that we should take in the project.  These actions will primarily influence the Data Management Plan.
\begin{itemize}
    \item (Pseudo) anonymisation.  Social media postings used in the project will remove personal identifiers, replacing with hash codes.
    \item Public figures.  We should be clear about how we treat public figures differently from private citizens (in particular with regard to  social media content), and should aim to provide a clear categorisation.
    \item Privacy by design. The design of the \project platform should ensure privacy is safeguarded.
    \item Legal by design.   The design of the \project platform should ensure legal compliance.
    \item Aggregation of personal data.  \project presentations of results should be ``aggregate views'' to avoid exposing personally identifiable opinions.
    \item Destruction of personal data.  Any personally identifiable information not needed should be destroyed;  however, this must also be balanced against the responsibility of the consortium to conduct reproducible research and the project goal of knowledge-base construction.  
    \item Images and videos in social media.  The consortium needs to decide how to proceed with images and videos contained in social media posts; the risk here is that individuals might be identifiable through the images / videos embedded in their posts.
 \end{itemize}
 