%This report will describe the procedures that will be implemented to ensure that the ethical standards and guidelines of Horizon2020 will be rigorously applied.

%Particular attention should be paid to the collection and processing of data from social media. Justification for the collection should be provided and safeguards implemented (including anonymization and data minimization). In a suitable task and deliverable in WP2, please include a justification for the data collection, and address the safeguards. The consortium must confirm that data collection and management are state of the art as far as privacy is concerned and that no private/sensitive information will be divulged by the use of the planned system (implementation of privacy by design). The nature of the use case (external media monitoring) contents is unclear and might raise some issues regarding processing broadcast media and internet content From Ukraine, Iran, Russia, and a number of Arab countries.

% The Ethics Committee will comprise a representative from each partner and two independent ethics advisors. It will meet as required, at least once per year. Any member of the Ethics Committee or Project Board may call a meeting to address a pressing problem. Meetings may be held face-to-face or virtually. The Ethics Committee is responsible for ethical issues arising from the project’s activities. It will:
% • Monitor usage and distribution of recordings and personal data;
% • In all experiments, ensure that participant rights are respected and that relevant national and EU legislation is
% followed.
% • Ensure that relevant experimental protocols used in the project receive approval from the relevant institutional
% ethical review committees.
% • Review potential increase in the ability to track social media and its potential use for military and intelligence
% use as a result of project advances, and recommend mitigation strategies.




\section{Introduction}

The goal of \project is the provision of media monitoring capabilities based on data from publicly available media streams. We will use this data to process, track and profile information about people and organisations, which means that the project needs to carefully address ethical issues relating to privacy.

We take the draft Data Science Ethical Framework  prepared by the UK Cabinet Office\footnote{\url{https://www.gov.uk/government/publications/data-science-ethical-framework}} as our guiding principles.  The main guidance is to ensure that \emph{the public benefit of the project is balanced against the risks of pursuing the project}.  The six principles of the framework are:
\begin{enumerate}
    \item Start with clear user need and public benefit
    \item Use data and tools which have the minimum intrusion necessary
    \item Create robust data science models
    \item Be alert to public perceptions
    \item Be as open as possible
    \item Keep data secure
\end{enumerate}

In this deliverable we first outline the ethical review and recommendation process that we have established in the \project project in Section~\ref{sec:process}.  We have grouped the ethical issues which arise into three broad categories:

\begin{itemize}
    \item Protection of personal data (discussed in Section~\ref{sec:personal}).  This includes the justification for collection of data potentially containing personal data, preservation of privacy, a privacy impact assessment, and links to the data management plan (including privacy by design).
    
    \item Import and export of data to non-EU countries (Section~\ref{sec:nonEU}).
    
    \item Broader ethical concerns including dual-use  (military and defence) implications of \project technologies, and the social impact of the tools and technologies developed by \project (Section~\ref{sec:broader}).
\end{itemize}

